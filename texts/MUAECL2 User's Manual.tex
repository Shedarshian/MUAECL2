\documentclass[UTF8]{ctexart}

\usepackage{url}
\usepackage{comment}
\usepackage{float}
\usepackage{fontspec}
\usepackage{fancyhdr}
\usepackage{listings}
\usepackage[colorlinks,
            linkcolor=blue,
            anchorcolor=black,
            citecolor=blue
            ]{hyperref}
\usepackage[top=0.7in, bottom=0.7in, left=1.2in, right=1.2in]{geometry}
\usepackage{makecell}
\usepackage{multirow}
\usepackage{array}
\usepackage{fancyvrb}

\author{MUA Group(Modification Ultraman Alliance)}
\title{MUAECL2 User's Manual  ver. 2.0.1\_alpha}
\date{\today}

\DefineVerbatimEnvironment{MUAvbt}{Verbatim}{frame=single, rulecolor=\color{magenta}}

\begin{document}
\maketitle
\tableofcontents
\clearpage

\section{关于}

MUAECL是MUA制作组制作的ecl编译器,可以将一种比较类似C的语言——MUAECL语言的mecl文本文件,编译成ecl文件,并且可以被MUA引擎,以及th15的引擎所识别运行。

\section{安装与运行}

下载后解压至任意文件夹即可。解压后的文件包括:

\begin{table}[H]
	\centering
	\begin{tabular}{c|c}
		\hline
		文件名 & 用途 \\\hline
		MUAECL2.exe & 程序本体 \\\hline
		action.csv & 跳转表 \\\hline
		ins.ini & ins对照表 \\\hline
		default.ini & default.ecl中包含的子程序名 \\\hline
		include.ini & 预定义预处理指令 \\\hline
	\end{tabular}
\end{table}

这些文件需要被放在同一文件夹下。

本程序使用命令行方式运行。在cmd中运行``\verb|MUAECL2.x.x.exe -h|''(实际exe名随版本号而不同)或``\verb|MUAECL2.x.x.exe --help|''即可查看命令行帮助。

\begin{table}[H]
	\centering
	\begin{tabular}{l|l|c}
		\hline
		\multicolumn{2}{c|}{命令行参数} & 用途 \\\hline
		-h & --help & 显示命令行帮助 \\\hline
		-c & --compile & 将MUAECL2源代码编译成ecl文件 \\\hline
		-p & --preprocess & 预处理MUAECL2源代码,但不编译 \\\hline
		-i <filename> & --input-file <filename> & 输入文件名 \\\hline
		-o <filename> & --output-file <filename> & 输出文件名 \\\hline
		-n & -no-preprocess & 在编译时不进行预处理 \\\hline
	\end{tabular}
\end{table}

命令行应用示例:

\begin{MUAvbt}
MUAECL2 -h
MUAECL2 -c -i st01.mecl -o st01.ecl
MUAECL2 -p -i st01.mecl -o st01p.mecl
MUAECL2 -c -n -i st01p.mecl -o st01.ecl
\end{MUAvbt}

\section{MUAECL语言}

MUAECL语言是一种类似C的,描述MUA引擎中诸如子弹发射、敌人出场等信息的脚本。

\subsection{预处理指令}

预处理时,MUAECL首先检测每行行尾的反斜杠,对于行尾有反斜杠的行,将反斜杠删除,并将下一行的文本移至本行。

MUAECL的预处理指令由\verb|#|开始,共有以下几种:

\paragraph{s指令(s代表substitute)} \verb|#s| \textit{deliminator pattern deliminator string deliminator name}

作用:将后续文本中每行内的\textit{pattern}的所有匹配(识别\href{https://en.wikipedia.org/wiki/Regular_expression}{正则表达式})替换成\textit{string},直到\textit{name}相同的\verb|ends|指令出现为止。若没有对应的ends指令,则替换直到文件尾。

示例:

\begin{MUAvbt}
#s/abc/def/name
#s ((?:a|b|c)*)d(e?) $2$1 mysubs1
\end{MUAvbt}

\begin{enumerate}
	\item \textit{deliminator}为一个任意非标识符字符\verb|[^a-zA-Z0-9_]|。
	\item \textit{pattern}与\textit{string}内的特殊字符依照\href{https://zh.cppreference.com/w/cpp/regex/ecmascript}{ECMAScript}规则(即C++标准库<reg\_ex>中默认的语法规则)替换
	\item \textit{pattern}与\textit{string}按照全部字面意思读取,包括空格与制表符,但不包括换行符。
	\item \textit{pattern}与\textit{string}内不能包含该非标识符字符,也不能跨行匹配或替换为多行文本。
	\item \textit{name}不得包含非标识符字符。
	\item 若有两个\textit{name}相同则报错
\end{enumerate}

\paragraph{无名s指令} \verb|#s| \textit{deliminator pattern deliminator string deliminator}

作用:将后续文本中\textit{pattern}的所有匹配替换成\textit{string},直到文件尾。

\paragraph{ends指令} \verb|#ends| \textit{deliminator name}

作用:结束名为\textit{name}的替换。\textit{deliminator}与\textit{name}的要求与\verb|s|指令相同。

\paragraph{define指令} \verb|#define| \textit{space identifier space string}

作用:与c相同,将\textit{identifier}关键词替换成\textit{string}。

示例:\verb|#define abc def| \\ 会将文本中\verb|abc + abcd|替换成\verb|def + abcd|。

\begin{enumerate}
	\item 相当于匹配``\verb|\b|\textit{identifier}\verb|\b|'',并且\textit{name}定义为\textit{identifier}的\verb|s|指令。
	\item \textit{identifier}不得包含非标识符字符。
\end{enumerate}

\paragraph{函数式define指令} \verb|#define| \textit{space identifier\{paraname1, paraname2, ...\} space string}

作用:与c类似,将带参数的\textit{identifier}关键词替换成\textit{string},与c不同的是,对于\textit{string}中\textit{paraname}的处理,只替换字符串不识别标识符。\textbf{所以请一定要注意各个paraname不要有相互包含的字符串,而且也不要使用string中包含的}。

示例:\verb|#define abc{d,e,f} (d+e+f+def)| \\ 会将文本中\verb|abc{1,3,5}|替换成\verb|(1+3+5+135)|。

\begin{enumerate}
	\item 相当于匹配``\verb|\b|\textit{identifier}\verb|\b|'',并且\textit{name}定义为\textit{identifier}的s指令。
	\item \textit{identifier}与\textit{paraname}不得包含非标识符字符。
	\item 以示例为例,等价于:将\textit{string}中\textit{paraname1}替换成'\verb|$01|',将\textit{paraname2}替换成'\verb|$02|',以此类推后:\\
	\verb|#s \babc\{\s*(.*?\S)\s*,\s*(.*?\S)\s*,\s*(.*?\S)\s*\} ($01+$02+$03+$01$02$03) abc|
\end{enumerate}

\paragraph{undef指令} \verb|#undef| \textit{space name}

结束\textit{name}的后续匹配。等价于\verb|#ends|\ \textit{name}。

\paragraph{ecli包含} \verb|#ecli| \textit{space eclfile}
\paragraph{anim包含} \verb|#anim| \textit{space anmfile}

为文件添加ecl文件与anm文件的包含。

示例:\verb|#ecli default.ecl|

预处理不改变后续文本的行号。

\subsection{预定义宏变量}

mecl包含几个预定义宏变量:

\begin{table}[H]
	\centering
	\begin{tabular}{c|c}
		\hline
		宏名 & 作用 \\\hline
		\verb|__FILE__| & 展开成当前文件名 \\\hline
		\verb|__DATE__| & 展开成翻译到当前关键字时的日期 \\\hline
		\verb|__TIME__| & 展开成翻译到当前关键字时的时间 \\\hline
		\verb|__LINE__| & 展开成行号 \\\hline
		\verb|pi| & 展开成3.14159265 \\\hline
	\end{tabular}
\end{table}

\section{基本概念}

\subsection{注释}

mecl允许两种注释:C风格注释或C++风格注释:

\begin{MUAvbt}
1. /* some content */
2. // some content \n
\end{MUAvbt}

与C不同的是,注释的处理在预处理替换步骤之后,也即:1.注释范围内的预处理指令会被处理;2.预处理替换后组合而成的注释符号会被识别。

\subsection{标识符}

与C相同,一个\textit{标识符}是一个由数字,下划线,小写和大写拉丁字母组成的任意长度的序列,且不可以数字起始。标识符区分大小写。

有效的标识符示例:\verb|shedarshian|,\verb|Temp0_float|,\verb|_name_|

\subsubsection{全局变量}

mecl语言中包含若干全局变量,诸如自机坐标、内置随机数等,列表可在ins.ini中查询。若标识符与某一全局变量名相同,则会被识别成该全局变量。

\subsection{类型}

mecl语言包含如下几种类型:

void类型,int类型,float类型,string类型,point类型。(目前版本中对于string类型\textbf{变量}尚无支持)

mecl中int与float类型占据4字节。

称算术类型包含int,float,point,基本算术类型包含int,float

\subsubsection{隐式类型转换}

在需要时(如\hyperref[chongzai]{重载决议}发生时,或if、for表达式内),会检测是否需要隐式类型转换。

隐式类型转换包括:左值到右值转换,整数到浮点数转换,以及void类型左值到任意类型左值转换。

\section{子程序}

mecl由多个子程序(\verb|sub|)组成:\textit{suffix} \verb|sub| \textit{subname} \verb|(| \textit{subvars} \verb|)| \textit{suffix} \verb|{| \textit{subcontents} \verb|}|

\textit{suffix}为可选的\verb|no_overload|关键字。参数列表为\textit{参数声明}的逗号分隔列表,每一项包含一个类型名以及一个变量名。这些具名形参均可在函数体内被访问。函数体\textit{subcontents}内为一条条的\hyperref[yuju]{语句}。

示例:

\begin{MUAvbt}
sub MainSub01(float x) no_overload
{
	//some content
}
no_overload sub Main()
{
	//some content
}
sub Boss_at1(int num, float angle) {
	//some content
}
\end{MUAvbt}

子程序之间允许同名(\textbf{重载}),只要参数数或类型不同。调用重载的函数时遵循\hyperref[chongzai]{重载决议}(后述)。

默认情况下,编译器输出ecl文件时会使用输入参数类型名装饰sub名。\verb|no_overload|关键字指示编译器在输出至ecl文件时不装饰sub名,同时禁止了对此函数的重载。

子程序名为全局,也即允许\textbf{后声明先调用}。

\section{表达式}

表达式是\textit{运算符}和它们的\textit{运算数}的序列,它指定一项计算。

\subsection{类型与值类型}

每个表达式有一个类型,和一个值类型。值类型为左值或右值。左值指示其求值确定一个对象个体,右值指示其求值没有结果对象。

下列表达式是左值表达式:
\begin{enumerate}
	\item 变量名\verb|a|。
	\item 赋值及复合赋值表达式,如\verb|a = b|。
	\item 间接寻址表达式,如\verb|*p|。
\end{enumerate}

左值可以取地址\verb|&(i = 1)|,左值可以作为赋值运算符的左操作数。

下列表达式是右值表达式:
\begin{enumerate}
	\item 字面量,如\verb|42|。
	\item 函数调用表达式,如\verb|g(1)|。
	\item 算术表达式、逻辑表达式、比较表达式,如\verb|a + 1|、\verb|c && d|等。
	\item 取地址表达式,如\verb|&a|。
	\item 转型表达式,如\verb|(int)f|。
\end{enumerate}

右值不能被取地址,右值不能作为赋值运算符的左操作数。

\subsection{初等表达式}

初等表达式包括字面量与标识符。

\paragraph{整数字面量} 整数字面量包含以下几种:

\begin{enumerate}
	\item 十进制字面量,由\verb|1-9|的数位后随零或多个\verb|0-9|的数位组成;
	\item 八进制字面量,由\verb|0|后随多个\verb|0-7|的数位组成;
	\item 十六进制字面量,由\verb|0x|后随多个\verb|0-9|或\verb|A-F|或\verb|a-f|组成。
\end{enumerate}

例如下列变量被初始化到相同值:

\begin{MUAvbt}
int d = 42;
int o = 052;
int x = 0x2a;
int X = 0x2A;
\end{MUAvbt}

\paragraph{浮点数字面量} 浮点数字面量为一个包含且仅包含一个小数点的十进制数,后随一个可选的\verb|f|组成的字面量。如:\verb|2.34|,\verb|.1f|,\verb|1.|。

\paragraph{字符串字面量} 字符串字面量为用一对双引号\verb|""|围起来的字符序列。反斜杠可以用于escape,在单个反斜杠后的字符以本义被录入,可用于在字符串中添加双引号字符本身。字符串字面量拥有string类型。

\paragraph{预定义常量} 预定义常量在ins.ini中定义,被转义成等效的整数字面量或浮点数字面量。支持的预定义常量将在对应的instruction处列出。

\paragraph{标识符} 标识符可以用于函数名,或者指代一个变量。%未来可能支持成员名

\subsection{运算符}

运算符有以下几类:

\begin{table}[H]
	\centering
	\begin{tabular}{|c|c|c|c|c|c|}
		\hline
		\hyperref[fuzhi]{赋值} & \hyperref[suanshu]{算术} & \hyperref[luoji]{逻辑} & \hyperref[bijiao]{比较} & \hyperref[chengyuan]{成员访问} & \hyperref[qita]{其他} \\\hline
		% \begin{comment}
\begin{minipage}{0.6in}
\begin{verbatim}

a = b
a += b
a -= b
a *= b
a /= b
a %= b
a &&= b
a ||= b
a &= b
a |= b
a ^= b

\end{verbatim}
\end{minipage}
&
\begin{minipage}{0.5in}
\begin{verbatim}
-a
a + b
a - b
a * b
a / b
a % b
a & b
a | b
a ^ b
\end{verbatim}
\end{minipage}
&
\begin{minipage}{0.6in}
\begin{verbatim}
!a
a && b
a || b
a and b
a or b
\end{verbatim}
\end{minipage}
&
\begin{minipage}{0.5in}
\begin{verbatim}
a == b
a != b
a < b
a <= b
a > b
a >= b
\end{verbatim}
\end{minipage}
&
\begin{minipage}{0.4in}
\begin{verbatim}
a.b
a[b]
*a
&a
\end{verbatim}
\end{minipage}
&
\begin{minipage}{1.3in}
\begin{verbatim}
id(...)
a1 : a2 : a3 : a4
(type)a
(a)
\end{verbatim}
\end{minipage}
\\\hline
% \end{comment}
	\end{tabular}
\end{table}

内建运算符有几种内建的重载,\hyperref[chongzai]{重载决议}规则决定调用哪个重载。

\subsubsection{算术运算符}
\label{suanshu}

\paragraph{一元负运算符} 返回运算数的相反数。形式为:\verb|-expr|。

对于算术类型\verb|A|,下列函数签名参与\hyperref[chongzai]{重载决议}:%算术类型:href
\begin{MUAvbt}
A operator-(A)
\end{MUAvbt}

\paragraph{加减运算符} 返回运算数之和或差。形式为:\verb|expr + expr|,\verb|expr - expr|。

对于算术类型\verb|A|,下列函数签名参与\hyperref[chongzai]{重载决议}:
\begin{MUAvbt}
A operator+(A, A)
A operator-(A, A)
string operator+(string, string)
\end{MUAvbt}

\paragraph{乘法性运算符} 返回运算数之积、商或余数。对于整数之商,向零截断小数部分。形式为:\verb|expr * expr|,\verb|expr / expr|,\verb|expr % expr|

对于基本算术类型\verb|I|,下列函数签名参与\hyperref[chongzai]{重载决议}:
\begin{MUAvbt}
I operator*(I, I)
I operator/(I, I)
int operator%(int, int)
point operator*(float, point)
point operator*(point, float)
point operator/(point, float)
\end{MUAvbt}

\paragraph{按位逻辑运算符} 返回运算数之按位与、按位或、按位异或。形式为:\verb|expr & expr|,\verb&expr | expr&,\verb|expr ^ expr|。

下列函数签名参与\hyperref[chongzai]{重载决议}:
\begin{MUAvbt}
int operator&(int, int)
int operator|(int, int)
int operator^(int, int)
\end{MUAvbt}

\subsubsection{逻辑运算符}
\label{luoji}

逻辑运算符返回运算数之逻辑与、逻辑或、逻辑非。此处以int代替布尔值,非0为真,0为假。返回值以1代表真,0代表假。形式为:\verb|expr && expr|,\verb&expr || expr&,\verb|expr and expr|,\verb&expr or expr&,\verb|! expr|。提供两种版本的与、或运算符:符号形式短路求值,单词形式不短路求值。短路求值即:对于\verb|&&|,若第一运算数为假则不求值第二运算数;对于\verb&||&,若第一运算数为真则不求值第二运算数。

下列函数签名参与\hyperref[chongzai]{重载决议}:
\begin{MUAvbt}
int operator&&(int, int)
int operator||(int, int)
int operator and(int, int)
int operator or(int, int)
int operator!(int)
\end{MUAvbt}

\subsubsection{比较运算符}
\label{bijiao}

比较运算符返回运算数间是否相等以及大小关系。形式为:\verb|expr == expr|,\verb|expr != expr|,\verb|expr < expr|,\verb|expr <= expr|,\verb|expr > expr|,\verb|expr >= expr|。

对于全部四种类型之一\verb|T|(int, float, point, string),下列函数签名参与\hyperref[chongzai]{重载决议}:
\begin{MUAvbt}
int operator==(T, T)
int operator!=(T, T)
int operator<(T, T)
int operator<=(T, T)
int operator>(T, T)
int operator>=(T, T)
\end{MUAvbt}

\subsubsection{赋值运算符}
\label{fuzhi}

赋值运算符的右操作数允许直接使用冒号分隔表达式,其余运算符则不允许。赋值运算符期待可修改左值为其左运算数,右值表达式为其右运算数。

\paragraph{简单赋值运算符} 简单赋值运算符以右操作数的值修改左操作数。形式为:\verb|expr = exprf|。赋值运算符返回左操作数的左值。

对于全部四种类型\verb|T|(int, float, point, string),下列函数签名参与\hyperref[chongzai]{重载决议}:
\begin{MUAvbt}
T& operator=(T&, T)
\end{MUAvbt}

\paragraph{复合赋值运算符} 复合赋值运算符以左操作数与右操作数运算后的值修改左操作数,\verb|E1 op= E2|的行为与表达式\verb|E1 = E1 op E2|行为相同,除了只求值表达式\verb|E1|一次。形式为:\verb|expr += exprf|,\verb|expr -= exprf|,\verb|expr *= exprf|,\verb|expr /= exprf|,\verb|expr %= exprf|,\verb|expr &&= exprf|,\verb&expr ||= exprf&,\verb|expr &= exprf|,\verb&expr |= exprf&,\verb|expr ^= exprf|。其中逻辑与赋值运算符、逻辑或赋值运算符有短路求值。

对于算术类型\verb|A|、基本算术类型\verb|I|、全部四种类型之一\verb|T|(int, float, point, string),下列函数签名参与\hyperref[chongzai]{重载决议}:
\begin{MUAvbt}
T& operator+=(T&, T)
A& operator-=(A&, A)
I& operator*=(I&, I)
point& operator*=(point&, float)
I& operator/=(I&, I)
point& operator/=(point&, float)
int& operator%=(int&, int)
int& operator&&=(int&, int)
int& operator||=(int&, int)
int& operator&=(int&, int)
int& operator|=(int&, int)
int& operator^=(int&, int)
\end{MUAvbt}

\subsubsection{成员访问运算符}
\label{chengyuan}

\paragraph{间接寻址运算符} 间接寻址运算符取出指针指向的操作数。形式为:\verb|*expr|。由于当前版本无指针类型,以int代替指针,故间接寻址运算符返回void左值对象,需使用类型转换表达式指定取出对象的类型。特别的是,此void左值对象可以隐式转换到任意类型,故可以利用此特性靠上下文推断出取出的对象类型。而不能推断出类型的表达式则视为错误。更多信息请参考\hyperref[chongzai]{重载决议}。

下列函数签名参与\hyperref[chongzai]{重载决议}:
\begin{MUAvbt}
void& operator*(int)
\end{MUAvbt}

\paragraph{取地址运算符} 取地址运算符获得指向该对象的指针。形式为:\verb|&expr|。由于当前版本无指针类型,故取地址运算符返回int类型。

对于全部四种类型之一\verb|T|(int, float, point, string)下列函数签名参与\hyperref[chongzai]{重载决议}:
\begin{MUAvbt}
int operator&(T&)
\end{MUAvbt}

\paragraph{成员运算符} \verb|expr.expr|,当前版本未加载其作用。

\paragraph{数组下标运算符} \verb|expr[expr]|,当前版本未加载其作用。

\subsubsection{其他运算符}
\label{qita}

\paragraph{函数调用运算符} 形式为:\verb|id(exprf, exprf, ...)|。\verb|id|为标识函数名的标识符,可以为用户定义的sub名或是内建的ins名或mode名。函数传入的参数列表可以为冒号分隔表达式或者普通表达式。用户定义的sub名可以重载,\hyperref[chongzai]{重载决议}规则决定调用哪个重载。需注意的是,几乎所有内建的ins与mode均返回void。

当前版本所有函数返回void右值。

\paragraph{冒号分隔表达式} 冒号分隔表达式是一种简便的难度switch,接受对应ENHL四种难度的四个参数,形式为\verb|expr : expr : expr : expr|。只有赋值表达式的右侧与函数调用支持冒号分隔表达式。

对于全部四种类型之一\verb|T|(int, float, point, string)下列函数签名参与\hyperref[chongzai]{重载决议}:
\begin{MUAvbt}
T operator:(T, T, T, T)
\end{MUAvbt}

\paragraph{C风格转型表达式} 转型表达式可以将一种类型转换为另一种类型。形式为:\verb|(type)expr|。转型表达式无转换类型限制(???)。

\paragraph{括号表达式} 括号表达式用于指定运算顺序。形式为:\verb|(expr)|。

\subsection{运算符优先级}

下表列出mecl运算符的优先级和结合性。运算符从顶到底以降序列出。

\begin{table}[H]
	\centering
	\begin{tabular}{|c|c|c|c|}
		\hline
		优先级 & 运算符 & 描述 & 结合性 \\\hline
		1 & \verb|()| & 括号 & 左到右 \\\hline
		\multirow{2}{*}{2} & \verb|a.b| & 取成员 & \multirow{2}{*}{左到右} \\\cline{2-3}
		& \verb|a[b]| & 取下标 & \\\hline
		\multirow{3}{*}{3} & \verb|!| & 逻辑非 & \multirow{3}{*}{右到左}\\\cline{2-3}
		& \verb|*a &a| & 寻址与取址 & \\\cline{2-3}
		& \verb|-a| & 一元负 & \\\hline
		4 & \verb|a*b a/b a%b| & 乘法 除法 余数 & 左到右 \\\hline
		5 & \verb|a+b a-b| & 加法 减法 & 左到右 \\\hline
		\multirow{2}{*}{6} & \verb|a<b a<=b| & 小于 小于等于 & \multirow{2}{*}{左到右} \\\cline{2-3}
		& \verb|a>b a>=b| & 大于 大于等于 & \\\hline
		7 & \verb|a==b a!=b| & 等于 不等于 & 左到右 \\\hline
		8 & \verb|a&b| & 按位与 & \multirow{7}{*}{左到右} \\\cline{1-3}
		9 & \verb|a^b| & 按位异或 & \\\cline{1-3}
		10 & \verb&a|b& & 按位或 & \\\cline{1-3}
		11 & \verb|a and b| & 非短路逻辑与 & \\\cline{1-3}
		12 & \verb|a or b| & 非短路逻辑或 & \\\cline{1-3}
		13 & \verb|a&&b| & 短路逻辑与 & \\\cline{1-3}
		14 & \verb&a||b& & 短路逻辑或 & \\\hline
		\multirow{2}{*}{15} & \verb|= += -= *= /=| & \multirow{2}{*}{所有赋值} & \multirow{2}{*}{右到左} \\\cline{2-2}
		& \verb"%= &&= ||= &= ^= |=" &  & \\\hline
		16 & \verb|a1:a2:a3:a4| & 冒号分隔 & - \\\hline
	\end{tabular}
\end{table}

\subsection{重载决议}
\label{chongzai}

选取一个函数或运算符的重载时,使用如下规则:

对该函数或运算符的所有重载判断输入参数个数是否相等,以及能否将每个输入参数类型隐式转换至目标类型。之后在所有可以转换到的重载函数集中选取最优的。最优定义为:有高优先级的某转换的隐式转换优于无该转换的。隐式转换的优先级由高到低为:

\verb|void&|$\to$\verb|point|, \verb|void&|$\to$\verb|string|, \verb|void&|$\to$\verb|float|, \verb|void&|$\to$\verb|int|, \verb|int|$\to$\verb|float|, \verb|lvalue|$\to$\verb|rvalue|.

\subsection{字面量优化}

编译器会对于字面量运算在编译时进行运算,优化后存入ecl脚本。如\verb|a*(1./2)|在ecl文件中会变成\verb|a*0.5|。

\section{语句}
\label{yuju}

mecl的每条语句(\textit{stmt})有以下几种类型:

\subsection{表达式语句}

跟随分号的表达式是语句。

\textit{expr} \verb|;|

示例:

\begin{MUAvbt}
a += 1;		//赋值语句
wait(20);		//ins调用
MainSub00();		//函数调用
a;		//弃值表达式
\end{MUAvbt}

\subsection{语句块}

\verb|{| \textit{stmts} \verb|}|

可以使用大括号括起多条语句,将其合并成一条。

\subsection{变量声明}

变量声明语句声明(并可选的初始化)一个或多个变量。形式为:\verb|type| \textit{vdecl} \verb|;|

其中\textit{vdecl}为一个或多个下列项的逗号分隔列表:\verb|id|或\verb|id =| \textit{inif}。

\textit{inif}可以为有或无冒号分隔的初始化列表或者表达式。

mecl的每条变量声明均为子程序局域,也即只要在同一个子程序中,允许\textbf{后声明先调用}。同样无局部作用域的概念。

示例:

\begin{MUAvbt}
int i;
float a = 1., b;
int j = 123 * 456 : 234 * 567 : 345 * 678 : 456 * 789;
\end{MUAvbt}

\subsubsection{初始化列表}

\verb|{| \textit{expr1, expr2,...} \verb|}|

初始化列表是用来初始化复杂算术类型(如point)的手段。示例:

\begin{MUAvbt}
point p = {1., 2.};
point p1 = {1., 2.}:{3., 4.}:{5., 6.}:{7., 8.};
\end{MUAvbt}

\subsection{选择语句}

选择语句在数个控制流中选择一个。形式为:

\begin{Verbatim}[frame=single, rulecolor=\color{magenta}, commandchars=\\\{\}]
if ( \textit{condition} ) \textit{true_stmt}
if ( \textit{condition} ) \textit{true_stmt} else \textit{false_stmt}
\end{Verbatim}

若\textit{condition}为非0则执行\textit{true\_stmt}(常为复合语句),否则执行\textit{false\_stmt}。

\textit{condition}需可以隐式转换至int右值。

在第二形式的\verb|if|语句(包含\verb|else|者)中,若\textit{true\_stmt}亦是\verb|if|语句,则内层\verb|if|语句必须也含有\verb|else|部分。换言之,嵌套\verb|if|语句中,\verb|else|关联到最近的尚未有\verb|else|的\verb|if|。

若通过\verb|goto|进入\textit{true\_stmt},则不执行\textit{false\_stmt}。

示例:

\begin{MUAvbt}
if (i == 1) {
	et_shape(0, ET_small, BLUE16);
	et_shoot(0);
	wait(30);
}
else if (i == 2)
	if (j == 2)
		wait(20);
	else			//此else与j == 2的if对应
		wait(30);
\end{MUAvbt}

\subsection{循环语句}

循环语句重复执行一些代码。形式为:

\begin{Verbatim}[frame=single, rulecolor=\color{magenta}, commandchars=\\\{\}]
while ( \textit{condition} ) \textit{stmt}
do \textit{stmt} while ( \textit{condition} ) ;
loop ( \textit{times} ) \textit{stmt}
\end{Verbatim}

在循环语句中可以使用break语句和continue语句进行\hyperref[break]{流程控制}。

\subsubsection{while循环}

while循环重复执行\textit{stmt},直到\textit{condition}变为0。

\textit{condition}需可以隐式转换至int右值。

示例:

\begin{MUAvbt}
float s = 10.;
while (s >= 0) {
	et_dir(0, RanDEG, 0.);
	et_speed(0, s, 0.);
	et_shoot(0);
	wait(1);
	s -= 0.1;
}
while (1)
	wait(1000);			//这是死循环
\end{MUAvbt}

\subsubsection{do-while循环}

do-while循环类似while循环,但至少执行一次\textit{stmt}。

示例:

\begin{MUAvbt}
float c = 1, s = 0;
do {
	s += c;
	c /= 2;
} while (c > 0.00001);
\end{MUAvbt}

\subsubsection{loop循环}

loop循环执行\textit{times}次\textit{stmt}。

\textit{times}需可以隐式转换至int右值。

示例:

\begin{MUAvbt}
loop (100) {
	et_shoot(1);
	wait(1);
}
\end{MUAvbt}

\subsection{标号语句}

标号语句建立一个标号,以供goto语句跳转。形式为:

\begin{Verbatim}[frame=single, rulecolor=\color{magenta}, commandchars=\\\{\}]
\textit{id} :
\end{Verbatim}

\subsection{跳转语句}

跳转语句无条件地转移控制流。形式为:

\begin{Verbatim}[frame=single, rulecolor=\color{magenta}, commandchars=\\\{\}]
break ;
continue ;
goto \textit{id} ;
\end{Verbatim}

\subsubsection{break语句}

break语句退出外围的while、do-while、loop语句。

break语句不能用于跳出多重嵌套循环。这种情况可以使用goto语句。

示例:

\begin{MUAvbt}
while (1) {
	et_shoot(0);
	wait(30);
	if (Yplayer <= 120.)
		break;
}
et_shoot(1);
\end{MUAvbt}

\subsubsection{continue语句}

continue语句跳过外围的while、do-while、loop语句中的剩余部分,继续下一次循环。如同用goto跳转到循环体尾。

示例:

\begin{MUAvbt}
loop (1000) {
	et_shoot(0);
	wait(10);
	if (Y_absol <= 224.)
		continue;
	et_shoot(1);
}
\end{MUAvbt}

\subsubsection{goto语句}

goto语句跳转至指定的标签处。goto语句必须在与它所用的标签相同的函数中,它可出现于标签前后。

\subsection{thread语句}

thread语句开启一个线程,与后续程序同步进行。形式为:

\begin{Verbatim}[frame=single, rulecolor=\color{magenta}, commandchars=\\\{\}]
thread \textit{id} ( \textit{exprf}, \textit{exprf}, ... ) ;
\end{Verbatim}

示例:

\begin{MUAvbt}
thread BossCard1_at1(1., 20, pi/20, RED32);
thread BossCard1_at1(2., 20, pi/20, GREEN32);
thread BossCard1_at1(1.5, 10, pi/10, CYAN32);
while (1) wait(1000);
\end{MUAvbt}

\subsection{rawins语句}

rawins语句使用数据直接构建ecl内容。形式为:

\begin{Verbatim}[frame=single, rulecolor=\color{magenta}, commandchars=\\\[\]]
__rawins {
	\textit[insdata] ;
	\textit[insdata] ;
	...
}
\end{Verbatim}

每条\textit{insdata}为:

\verb|{| \textit{id difficulty\_mask pop\_count param\_count param\_mask, param1 param2 ...} \verb|}|

\end{document}
