\documentclass{article}

\usepackage{url}
\usepackage{comment}
\usepackage{float}
\usepackage{fontspec}
\usepackage{fancyhdr}
\usepackage{listings}
\usepackage[colorlinks,
            linkcolor=blue,
            anchorcolor=black,
            citecolor=blue
            ]{hyperref}
\usepackage[top=0.7in, bottom=0.7in, left=1.2in, right=1.2in]{geometry}
\usepackage{makecell}
\usepackage{multirow}
\usepackage{array}
\usepackage{fancyvrb}

\author{MUA Group(Modification Ultraman Alliance)}
\title{MUAECL2 User's Manual  ver. 2.0.0\_alpha}
\date{today}

\DefineVerbatimEnvironment{MUAvbt}{Verbatim}{frame=single, rulecolor=\color{magenta}}

\begin{document}
\maketitle
\tableofcontents
\clearpage

\section{ABOUT}

MUAECL is a ecl compiler developed by MUA Group. It can translate a C-like language (MUAECL language) to a ecl file, in order to be read by either MUA engine, or th15 engine.

\section{INSTALLATION AND RUNNING}

Download and unzip to any folder. The unzipped folder contains:

\begin{table}[H]
	\centering
	\begin{tabular}{c|c}
		\hline
		File name & Usage \\\hline
		MUAECL2.exe & the program \\\hline
		action.csv & jumping table \\\hline
		ins.ini & instruction translate table \\\hline
		default.ini & sub names contained in default.ecl \\\hline
		include.ini & predefined preprocess directive \\\hline
	\end{tabular}
\end{table}

These files need to be contained in the same folder.

MUAECL run in command line. Running ``\verb|MUAECL2.x.x.exe -h|''(The real exe name will change along with version number) in cmd or ``\verb|MUAECL2.x.x.exe --help|'' will check the command line syntax help.

\begin{table}[H]
	\centering
	\begin{tabular}{l|l|c}
		\hline
		\multicolumn{2}{c|}{Command Line Argument} & Usage \\\hline
		-h & --help & Display command line syntax help \\\hline
		-c & --compile & Compile an MUAECL2 source file to a raw ECl file \\\hline
		-p & --preprocess & Preprocess (and not compile) an MUAECL2 source file \\\hline
		-i <filename> & --input-file <filename> & Input filename \\\hline
		-o <filename> & --output-file <filename> & Output filename \\\hline
		-n & -no-preprocess & Bypass the preprocessing step during compiling \\\hline
	\end{tabular}
\end{table}

Examples:

\begin{MUAvbt}
MUAECL2 -h
MUAECL2 -c -i st01.mecl -o st01.ecl
MUAECL2 -p -i st01.mecl -o st01p.mecl
MUAECL2 -c -n -i st01p.mecl -o st01.ecl
\end{MUAvbt}

\section{MUAECL Language}

MUAECL Language is a C-like script language, which describes some information for MUA engine, such as bullet shooting or enemy entering.

\subsection{Preprocess directives}

First, the MUAECL preprocessor checks for backslash which appears at the end of a line. Delete those backslashes, and moves the text on the next line to th original line.

MUAECL preprocess directives begins with \verb|#|. It contains those following types:

\paragraph{s directive (s stands for substitute)} \verb|#s| \textit{deliminator pattern deliminator string deliminator name}

Usage: Substitute all matching \textit{pattern} (checks for \href{https://en.wikipedia.org/wiki/Regular_expression}{regular expression}) with \textit{string} in following texts, until a \verb|ends| directive with the same \textit{name}. If no matching \verb|ends|, substitute until EOF.

Examples:

\begin{MUAvbt}
#s/abc/def/name
#s ((?:a|b|c)*)d(e?) $2$1 mysubs1
\end{MUAvbt}

\begin{enumerate}
	\item \textit{deliminator} should be a non-identifier character \verb|[^a-zA-Z0-9_]|.
	\item Special character in \textit{pattern} and \textit{string} will be interpreted according to \href{https://en.cppreference.com/w/cpp/regex/ecmascript}{ECMAScript} grammar, which is the same as what C++ standard library <reg\_ex> use by default.
	\item Read \textit{pattern} and \textit{string} all by literal, including space and tab, but not including newline.
	\item \textit{pattern} and \textit{string} are not allowed to contain the non-identifier character used for \textit{deliminator}, and not allowed to match or substitute to multiline text.
	\item \textit{name} could not contain non-identifier character.
	\item Preprocessor will report an error when two \textit{name} being the same.
\end{enumerate}

\paragraph{unnamed s directive} \verb|#s| \textit{deliminator pattern deliminator string deliminator}

Usage: Substitute all matching \textit{pattern} with \textit{string} in following texts until EOF.

\paragraph{ends directive} \verb|#ends| \textit{deliminator name}

Usage: End the substituting whose name is \textit{name}. The constrain of \textit{deliminator} and \textit{name} are the same as \verb|s| directive.

\paragraph{define directive} \verb|#define| \textit{space identifier space string}

Usage: Just like C, substitute keyword \textit{identifier} to \textit{string}.

Example:\verb|#define abc def| \\ Will substitute \verb|abc + abcd| to \verb|def + abcd|.

\begin{enumerate}
	\item Same as \verb|s| directive who matches ``\verb|\b|\textit{identifier}\verb|\b|'', while \textit{name} defined to \textit{identifier}
	\item \textit{identifier} could not contain non-identifier character.
\end{enumerate}

\paragraph{function type define directive} \verb|#define| \textit{space identifier\{paraname1, paraname2, ...\} space string}

Usage: Just like C, substitute keyword \textit{identifier} with parameters to \textit{string}. What do not like C is that during processing \textit{paraname} within \textit{string}, only substitute string but not identifier. \textbf{So PLEASE be cautious, DO NOT let paranames contain each other as substrings, or let \textit{string} contain some of the paranames.}

Example: \verb|#define abc{d,e,f} (d+e+f+def)| \\ Will substitute \verb|abc{1,3,5}| to \verb|(1+3+5+135)|.

\begin{enumerate}
	\item \textit{identifier} and \textit{paraname} could not contain non-identifier character.
	\item Using the example above, it is the same as: substitute \textit{paraname1} to '\verb|$01|', \textit{paraname2} to '\verb|$02|', and so on in \textit{string}: \\
	\verb|#s \babc\{\s*(.*?\S)\s*,\s*(.*?\S)\s*,\s*(.*?\S)\s*\} ($01+$02+$03+$01$02$03) abc|
\end{enumerate}

\paragraph{undef directive} \verb|#undef| \textit{space name}

Usage: End the matching of \textit{name}. Same as \verb|#ends| \textit{name}.

\paragraph{ecli including} \verb|#ecli| \textit{space eclfile}
\paragraph{anim including} \verb|#anim| \textit{space anmfile}

Add include ecl and anm files.

Example: \verb|#ecli default.ecl|

All preprocess will not change line numbers afterwards.

\subsection{Predefined macro variables}

mecl includes some predefined macro variables:

\begin{table}[H]
	\centering
	\begin{tabular}{c|c}
		\hline
		Macro name & Usage \\\hline
		\verb|__FILE__| & Expands to the name of the current file \\\hline
		\verb|__DATE__| & Expands to the date when translating the current keyword \\\hline
		\verb|__TIME__| & Expands to the time when translating the current keyword \\\hline
		\verb|__LINE__| & Expands to the line number \\\hline
		\verb|pi| & Expands to 3.14159265 \\\hline
	\end{tabular}
\end{table}

\section{BASIC CONCEPTS}

\subsection{COMMENTS}

mecl allows two types of comments: C-style and C++-style:

\begin{MUAvbt}
1. /* some content */
2. // some content \n
\end{MUAvbt}

Unlike C, the processing of comment is later than the preprocess substituting, which means: 1. Preprocess directives will be checked inside the comment range; 2. Comment token which is formed by preprocess substituting will be checked.

\subsection{IDENTIFIER}

Just like C, a \textit{identifier} is an arbitrarily long sequence of digits, underscores, lowercase and uppercase Latin letters, and should not begin with digits. Identifiers are case-sensitive.

Examples of valid identifiers: \verb|shedarshian|,\verb|Temp0_float|,\verb|_name_|

\subsubsection{GLOBAL VARIABLES}

mecl contains many global variables, such as the coordinate of player, the internal random number generator. A list will be found in ins.ini. If an identifier being the same as a global variable, it will be considered as that global variable.

\subsection{TYPES}

mecl contains types below:

void, int, float, string, point. (Current version do not support string type \textbf{VARIABLES}.)

In mecl, int and float occupies 4 bytes.

We call ``int, float, point'' as arithmetic type, and ``int float'' as fundamental arithmetic type.

\subsubsection{IMPLICIT TYPE CONVERSION}

Implicit type conversion will be check when needed (for example, when \hyperref[chongzai]{overload resolution} happens, or in if/for expressions).

Implicit conversion includes: lvalue to rvalue conversion; int to float conversion, and void lvalue to any type lvalue conversion.

\section{SUB-PROGRAMS}

mecl is made up of serveral sub-programs (\verb|sub|): \textit{suffix} \verb|sub| \textit{subname} \verb|(| \textit{subvars} \verb|)| \textit{suffix} \verb|{| \textit{subcontents} \verb|}|

\textit{suffix} is an optional \verb|no_overload| keyword. Parameter lists are comma-separated list of \textit{parameter declaration}, each of which contains a type name and a variable name. These named formal parameters can be approached inside function body. Function body \textit{subcontents} contains serveral \hyperref[yuju]{statements}.

Example:

\begin{MUAvbt}
sub MainSub01(float x) no_overload
{
	//some content
}
no_overload sub Main()
{
	//some content
}
sub Boss_at1(int num, float angle) {
	//some content
}
\end{MUAvbt}

Sub-programs are allowed to have same names (\textbf{overload}), as long as parameter number or type differs. \hyperref[chongzai]{Overload resolution} will take place when calling overloaded function.

By default, the compiler will use the parameter types to decorate the sub name. Keyword \verb|no_overload| will tell the compiler not to decorate the sub name, and in result, forbid the overload of the current function.

The sub-program name is global, which means \textbf{usage before declaration} is allowed.

\section{EXPRESSIONS}

An expression is a sequence of \textit{operators} and their \textit{operands}, that specifies a computation.

\subsection{TYPES AND VALUE CATEGORIES}

Each expression is characterized by two independent properties: a type and a value category. There are two value categories: lvalue and rvalue. Lvalue indicates its evaluation determines the identity of an object, while rvalue indicates its evaluation do not produce a long-lived object.

The following expressions are lvalue expressions:
\begin{enumerate}
	\item The name of a variable.
	\item Assignment and compound assignment expressions, such as \verb|a = b|.
	\item Indirection expressions, such as \verb|*p|.
\end{enumerate}

Lvalue's address may be taken\verb|&(i = 1)|, and lvalue may be used as left-hand operand of the assignment and compound assignment operators.

The following expressions are rvalue expressions:
\begin{enumerate}
	\item Literals, such as \verb|72|.
	\item Function call expressions, such as \verb|g(1)|.
	\item Arithmetic expressions, logical expressions, or comparision expressions, such as \verb|a + 1|、\verb|c && d|.
	\item Address-of expressions, such as \verb|&a|.
	\item Type cast expressions, such as \verb|(int)f|.
\end{enumerate}

Rvalue's address may not be taken, and rvalue can't be used as left-hand operand of the assignment and compound assignment operators.

\subsection{PRIMARY EXPRESSIONS}

Primary expressions includes literals and identifiers.

\paragraph{Integer literals} Integer literals contains:

\begin{enumerate}
	\item Decimal literals, which is a non-zero decimal digit \verb|1-9| followed by zero or more decimal digits \verb|0-9|;
	\item Octal literals, which is the digit \verb|0| followed by zero or more octal digits \verb|0-7|;
	\item Hex literals, which is the character sequence \verb|0x| followed by zero or more hexadecimal digits \verb|0-9| or \verb|A-F| or \verb|a-f|.
\end{enumerate}

For example, the following variables are initialized to the same value:

\begin{MUAvbt}
int d = 42;
int o = 052;
int x = 0x2a;
int X = 0x2A;
\end{MUAvbt}

\paragraph{Floating-point literal} Floating-point literal is a decimal number which contains and only contains one decimal separator, and a optional \verb|f| in the end, such as \verb|2.34|, \verb|.1f|, \verb|1.|.

\paragraph{String literal} String literal is a character sequence surrounded by a pair of double-quote \verb|""|. The backslash can be used to escape characters, since the character followed by a backslash will be recognized as itself. So it can be used to escape the double-quote character itself. String literals are of type string.

\paragraph{Predefined constant} Predefined constant are defined in ins.ini, and will be converted to corresponding integer or float literal. Predefined constant supported will be listed around corresponding instruction.

\paragraph{Identifier} Identifier can be used as function name, or to refer a variable.%May support point.x y in future

\subsection{运算符}

运算符有以下几类:

\begin{table}[H]
	\centering
	\begin{tabular}{|c|c|c|c|c|c|}
		\hline
		\hyperref[fuzhi]{赋值} & \hyperref[suanshu]{算术} & \hyperref[luoji]{逻辑} & \hyperref[bijiao]{比较} & \hyperref[chengyuan]{成员访问} & \hyperref[qita]{其他} \\\hline
		\begin{BVerbatim}{baseline=c}
a = b
a += b
a -= b
a *= b
a /= b
a %= b
a &&= b
a ||= b
a &= b
a |= b
a ^= b
\end{BVerbatim} &
\begin{BVerbatim}{baseline=c}
-a
a + b
a - b
a * b
a / b
a % b
a & b
a | b
a ^ b
\end{BVerbatim} &
\begin{BVerbatim}{baseline=c}
!a
a && b
a || b
a and b
a or b
\end{BVerbatim} &
\begin{BVerbatim}{baseline=c}
a == b
a != b
a < b
a <= b
a > b
a >= b
\end{BVerbatim} &
\begin{BVerbatim}{baseline=c}
a.b
a[b]
*a
&a
\end{BVerbatim} &
\begin{BVerbatim}{baseline=c}
id(...)
a1 : a2 : a3 : a4
(type)a
(a)
\end{BVerbatim} \\\hline
	\end{tabular}
\end{table}

内建运算符有几种内建的重载,\hyperref[chongzai]{重载决议}规则决定调用哪个重载。

\subsubsection{算术运算符}
\label{suanshu}

\paragraph{一元负运算符} 返回运算数的相反数。形式为:\verb|-expr|。

对于算术类型\verb|A|,下列函数签名参与\hyperref[chongzai]{重载决议}:%算术类型:href
\begin{MUAvbt}
A operator-(A)
\end{MUAvbt}

\paragraph{加减运算符} 返回运算数之和或差。形式为:\verb|expr + expr|,\verb|expr - expr|。

对于算术类型\verb|A|,下列函数签名参与\hyperref[chongzai]{重载决议}:
\begin{MUAvbt}
A operator+(A, A)
A operator-(A, A)
string operator+(string, string)
\end{MUAvbt}

\paragraph{乘法性运算符} 返回运算数之积、商或余数。对于整数之商,向零截断小数部分。形式为:\verb|expr * expr|,\verb|expr / expr|,\verb|expr % expr|

对于基本算术类型\verb|I|,下列函数签名参与\hyperref[chongzai]{重载决议}:
\begin{MUAvbt}
I operator*(I, I)
I operator/(I, I)
int operator%(int, int)
point operator*(float, point)
point operator*(point, float)
point operator/(point, float)
\end{MUAvbt}

\paragraph{按位逻辑运算符} 返回运算数之按位与、按位或、按位异或。形式为:\verb|expr & expr|,\verb&expr | expr&,\verb|expr ^ expr|。

下列函数签名参与\hyperref[chongzai]{重载决议}:
\begin{MUAvbt}
int operator&(int, int)
int operator|(int, int)
int operator^(int, int)
\end{MUAvbt}

\subsubsection{逻辑运算符}
\label{luoji}

逻辑运算符返回运算数之逻辑与、逻辑或、逻辑非。此处以int代替布尔值,非0为真,0为假。返回值以1代表真,0代表假。形式为:\verb|expr && expr|,\verb&expr || expr&,\verb|expr and expr|,\verb&expr or expr&,\verb|! expr|。提供两种版本的与、或运算符:符号形式短路求值,单词形式不短路求值。短路求值即:对于\verb|&&|,若第一运算数为假则不求值第二运算数;对于\verb&||&,若第一运算数为真则不求值第二运算数。

下列函数签名参与\hyperref[chongzai]{重载决议}:
\begin{MUAvbt}
int operator&&(int, int)
int operator||(int, int)
int operator and(int, int)
int operator or(int, int)
int operator!(int)
\end{MUAvbt}

\subsubsection{比较运算符}
\label{bijiao}

比较运算符返回运算数间是否相等以及大小关系。形式为:\verb|expr == expr|,\verb|expr != expr|,\verb|expr < expr|,\verb|expr <= expr|,\verb|expr > expr|,\verb|expr >= expr|。

对于全部四种类型之一\verb|T|(int, float, point, string),下列函数签名参与\hyperref[chongzai]{重载决议}:
\begin{MUAvbt}
int operator==(T, T)
int operator!=(T, T)
int operator<(T, T)
int operator<=(T, T)
int operator>(T, T)
int operator>=(T, T)
\end{MUAvbt}

\subsubsection{赋值运算符}
\label{fuzhi}

赋值运算符的右操作数允许直接使用冒号分隔表达式,其余运算符则不允许。赋值运算符期待可修改左值为其左运算数,右值表达式为其右运算数。

\paragraph{简单赋值运算符} 简单赋值运算符以右操作数的值修改左操作数。形式为:\verb|expr = exprf|。赋值运算符返回左操作数的左值。

对于全部四种类型\verb|T|(int, float, point, string),下列函数签名参与\hyperref[chongzai]{重载决议}:
\begin{MUAvbt}
T& operator=(T&, T)
\end{MUAvbt}

\paragraph{复合赋值运算符} 复合赋值运算符以左操作数与右操作数运算后的值修改左操作数,\verb|E1 op= E2|的行为与表达式\verb|E1 = E1 op E2|行为相同,除了只求值表达式\verb|E1|一次。形式为:\verb|expr += exprf|,\verb|expr -= exprf|,\verb|expr *= exprf|,\verb|expr /= exprf|,\verb|expr %= exprf|,\verb|expr &&= exprf|,\verb&expr ||= exprf&,\verb|expr &= exprf|,\verb&expr |= exprf&,\verb|expr ^= exprf|。其中逻辑与赋值运算符、逻辑或赋值运算符有短路求值。

对于算术类型\verb|A|、基本算术类型\verb|I|、全部四种类型之一\verb|T|(int, float, point, string),下列函数签名参与\hyperref[chongzai]{重载决议}:
\begin{MUAvbt}
T& operator+=(T&, T)
A& operator-=(A&, A)
I& operator*=(I&, I)
point& operator*=(point&, float)
I& operator/=(I&, I)
point& operator/=(point&, float)
int operator%=(int, int)
int operator&&=(int, int)
int operator||=(int, int)
int operator&=(int, int)
int operator|=(int, int)
int operator^=(int, int)
\end{MUAvbt}

\subsubsection{成员访问运算符}
\label{chengyuan}

\paragraph{间接寻址运算符} 间接寻址运算符取出指针指向的操作数。形式为:\verb|*expr|。由于当前版本无指针类型,以int代替指针,故间接寻址运算符返回void左值对象,需使用类型转换表达式指定取出对象的类型。特别的是,此void左值对象可以隐式转换到任意类型,故可以利用此特性靠上下文推断出取出的对象类型。而不能推断出类型的表达式则视为错误。更多信息请参考\hyperref[chongzai]{重载决议}。

下列函数签名参与\hyperref[chongzai]{重载决议}:
\begin{MUAvbt}
void& operator*(int)
\end{MUAvbt}

\paragraph{取地址运算符} 取地址运算符获得指向该对象的指针。形式为:\verb|&expr|。由于当前版本无指针类型,故取地址运算符返回int类型。

对于全部四种类型之一\verb|T|(int, float, point, string)下列函数签名参与\hyperref[chongzai]{重载决议}:
\begin{MUAvbt}
int operator&(T&)
\end{MUAvbt}

\paragraph{成员运算符} \verb|expr.expr|,当前版本未加载其作用。

\paragraph{数组下标运算符} \verb|expr[expr]|,当前版本未加载其作用。

\subsubsection{其他运算符}
\label{qita}

\paragraph{函数调用运算符} 形式为:\verb|id(exprf, exprf, ...)|。\verb|id|为标识函数名的标识符,可以为用户定义的sub名或是内建的ins名或mode名。函数传入的参数列表可以为冒号分隔表达式或者普通表达式。用户定义的sub名可以重载,\hyperref[chongzai]{重载决议}规则决定调用哪个重载。需注意的是,几乎所有内建的ins与mode均返回void。

当前版本所有函数返回void右值。

\paragraph{冒号分隔表达式} 冒号分隔表达式是一种简便的难度switch,接受对应ENHL四种难度的四个参数,形式为\verb|expr : expr : expr : expr|。只有赋值表达式的右侧与函数调用支持冒号分隔表达式。

对于全部四种类型之一\verb|T|(int, float, point, string)下列函数签名参与\hyperref[chongzai]{重载决议}:
\begin{MUAvbt}
T operator:(T, T, T, T)
\end{MUAvbt}

\paragraph{C风格转型表达式} 转型表达式可以将一种类型转换为另一种类型。形式为:\verb|(type)expr|。转型表达式无转换类型限制(???)。

\paragraph{括号表达式} 括号表达式用于指定运算顺序。形式为:\verb|(expr)|。

\subsection{运算符优先级}

下表列出mecl运算符的优先级和结合性。运算符从顶到底以降序列出。

\begin{table}[H]
	\centering
	\begin{tabular}{|c|c|c|c|}
		\hline
		优先级 & 运算符 & 描述 & 结合性 \\\hline
		1 & \verb|()| & 括号 & 左到右 \\\hline
		\multirow{2}{*}{2} & \verb|a.b| & 取成员 & \multirow{2}{*}{左到右} \\\cline{2-3}
		& \verb|a[b]| & 取下标 & \\\hline
		\multirow{3}{*}{3} & \verb|!| & 逻辑非 & \multirow{3}{*}{右到左}\\\cline{2-3}
		& \verb|*a &a| & 寻址与取址 & \\\cline{2-3}
		& \verb|-a| & 一元负 & \\\hline
		4 & \verb|a*b a/b a%b| & 乘法 除法 余数 & 左到右 \\\hline
		5 & \verb|a+b a-b| & 加法 减法 & 左到右 \\\hline
		\multirow{2}{*}{6} & \verb|a<b a<=b| & 小于 小于等于 & \multirow{2}{*}{左到右} \\\cline{2-3}
		& \verb|a>b a>=b| & 大于 大于等于 & \\\hline
		7 & \verb|a==b a!=b| & 等于 不等于 & 左到右 \\\hline
		8 & \verb|a&b| & 按位与 & \multirow{7}{*}{左到右} \\\cline{1-3}
		9 & \verb|a^b| & 按位异或 & \\\cline{1-3}
		10 & \verb&a|b& & 按位或 & \\\cline{1-3}
		11 & \verb|a and b| & 非短路逻辑与 & \\\cline{1-3}
		12 & \verb|a or b| & 非短路逻辑或 & \\\cline{1-3}
		13 & \verb|a&&b| & 短路逻辑与 & \\\cline{1-3}
		14 & \verb&a||b& & 短路逻辑或 & \\\hline
		\multirow{2}{*}{15} & \verb|= += -= *= /=| & \multirow{2}{*}{所有赋值} & \multirow{2}{*}{右到左} \\\cline{2}
		& \verb"%= &&= ||= &= ^= |=" &  & \\\hline
		16 & \verb|a1:a2:a3:a4| & 冒号分隔 & - \\\hline
	\end{tabular}
\end{table}

\subsection{重载决议}
\label{chongzai}

选取一个函数或运算符的重载时,使用如下规则:

对该函数或运算符的所有重载判断输入参数个数是否相等,以及能否将每个输入参数类型隐式转换至目标类型。之后在所有可以转换到的重载函数集中选取最优的。最优定义为:有高优先级的某转换的隐式转换优于无该转换的。隐式转换的优先级由高到低为:

void\&$\to$point, void\&$\to$string, void\&$\to$float, void\&$\to$int, int$\to$float, lvalue$\to$rvalue.

\subsection{字面量优化}

编译器会对于字面量运算在编译时进行运算,优化后存入ecl脚本。如\verb|a*(1./2)|在ecl文件中会变成\verb|a*0.5|。

\section{语句}
\label{yuju}

mecl的每条语句(\textit{stmt})有以下几种类型:

\section{表达式语句}

跟随分号的表达式是语句。

\textit{expr} \verb|;|

示例:

\begin{MUAvbt}
a += 1;			//赋值语句
wait(20);		//ins调用
a;				//弃值表达式
\end{MUAvbt}

\subsection{语句块}

\verb|{| \textit{stmts} \verb|}|

可以使用大括号括起多条语句,将其合并成一条。

\subsection{变量声明}

变量声明语句声明(并可选的初始化)一个或多个变量。形式为:\verb|type| \textit{vdecl} \verb|;|

其中\textit{vdecl}为一个或多个下列项的逗号分隔列表:\verb|id|或\verb|id =| \textit{inif}。

\textit{inif}可以为有或无冒号分隔的初始化列表或者表达式。

mecl的每条变量声明均为子程序局域,也即只要在同一个子程序中,允许\textbf{后声明先调用}。同样无局部作用域的概念。

示例:

\begin{MUAvbt}
int i;
float a = 1., b;
int j = 123 * 456 : 234 * 567 : 345 * 678 : 456 * 789;
\end{MUAvbt}

\subsubsection{初始化列表}

\verb|{| \textit{expr1, expr2,...} \verb|}|

初始化列表是用来初始化复杂算术类型(如point)的手段。示例:

\begin{MUAvbt}
point p = {1., 2.};
point p1 = {1., 2.}:{3., 4.}:{5., 6.}:{7., 8.};
\end{MUAvbt}

\subsection{选择语句}

选择语句在数个控制流中选择一个。形式为:

\begin{Verbatim}{frame=single, rulecolor=\color{magenta}, commandchars=\\\{\}}
if ( \textit{condition} ) \textit{true_stmt}
if ( \textit{condition} ) \textit{true_stmt} else \textit{false_stmt}
\end{Verbatim}

若\textit{condition}为非0则执行\textit{true_stmt}(常为复合语句),否则执行\textit{false_stmt}。

\textit{condition}需可以隐式转换至int右值。

在第二形式的\verb|if|语句(包含\verb|else|者)中,若\textit{true_stmt}亦是\verb|if|语句,则内层\verb|if|语句必须也含有\verb|else|部分。换言之,嵌套\verb|if|语句中,\verb|else|关联到最近的尚未有\verb|else|的\verb|if|。

若通过\verb|goto|进入\textit{true_stmt},则不执行\textit{false_stmt}。

示例:

\begin{MUAvbt}
if (i == 1) {
	et_shape(0, ET_small, BLUE16);
	et_shoot(0);
	wait(30);
}
else if (i == 2)
	if (j == 2)
		wait(20);
	else			//此else与j == 2的if对应
		wait(30);
\end{MUAvbt}

\subsection{循环语句}

循环语句重复执行一些代码。形式为:

\begin{Verbatim}{frame=single, rulecolor=\color{magenta}, commandchars=\\\{\}}
while ( \textit{condition} ) \textit{stmt}
do \textit{stmt} while ( \textit{condition} ) ;
loop ( \textit{times} ) \textit{stmt}
\end{Verbatim}

在循环语句中可以使用break语句和continue语句进行\hyperref[break]{流程控制}。

\subsubsection{while循环}

while循环重复执行\textit{stmt},直到\textit{condition}变为0。

\textit{condition}需可以隐式转换至int右值。

示例:

\begin{MUAvbt}
float s = 10.;
while (s >= 0) {
	et_dir(0, RanDEG, 0.);
	et_speed(0, s, 0.);
	et_shoot(0);
	wait(1);
	s -= 0.1;
}
while (1)
	wait(1000);			//这是死循环
\end{MUAvbt}

\subsubsection{do-while循环}

do-while循环类似while循环,但至少执行一次\textit{stmt}。

示例:

\begin{MUAvbt}
float c = 1, s = 0;
do {
	s += c;
	c /= 2;
} while (c > 0.00001);
\end{MUAvbt}

\subsubsection{loop循环}

loop循环执行\textit{times}次\textit{stmt}。

\textit{times}需可以隐式转换至int右值。

示例:

\begin{MUAvbt}
loop (100) {
	et_shoot(1);
	wait(1);
}
\end{MUAvbt}

\subsection{标号语句}

标号语句建立一个标号,以供goto语句跳转。形式为:

\begin{Verbatim}{frame=single, rulecolor=\color{magenta}, commandchars=\\\{\}}
\textit{id} :
\end{Verbatim}

\subsection{跳转语句}

跳转语句无条件地转移控制流。形式为:

\begin{Verbatim}{frame=single, rulecolor=\color{magenta}, commandchars=\\\{\}}
break ;
continue ;
goto \textit{id} ;
\end{Verbatim}

\subsubsection{break语句}

break语句退出外围的while、do-while、loop语句。

break语句不能用于跳出多重嵌套循环。这种情况可以使用goto语句。

示例:

\begin{MUAvbt}
while (1) {
	et_shoot(0);
	wait(30);
	if (Yplayer <= 120.)
		break;
}
et_shoot(1);
\end{MUAvbt}

\subsubsection{continue语句}

continue语句跳过外围的while、do-while、loop语句中的剩余部分,继续下一次循环。如同用goto跳转到循环体尾。

示例:

\begin{MUAvbt}
loop (1000) {
	et_shoot(0);
	wait(10);
	if (Y_absol <= 224.)
		continue;
	et_shoot(1);
}
\end{MUAvbt}

\subsubsection{goto语句}

goto语句跳转至指定的标签处。goto语句必须在与它所用的标签相同的函数中,它可出现于标签前后。

\subsection{thread语句}

thread语句开启一个线程,与后续程序同步进行。形式为:

\begin{Verbatim}{frame=single, rulecolor=\color{magenta}, commandchars=\\\{\}}
thread \textit{id} ( \textit{exprf}, \textit{exprf}, ... ) ;
\end{Verbatim}

示例:

\begin{MUAvbt}
thread BossCard1_at1(1., 20, pi/20, RED32);
thread BossCard1_at1(2., 20, pi/20, GREEN32);
thread BossCard1_at1(1.5, 10, pi/10, CYAN32);
while (1) wait(1000);
\end{MUAvbt}

\subsection{rawins语句}

rawins语句使用数据直接构建ecl内容。形式为:

\begin{Verbatim}{frame=single, rulecolor=\color{magenta}, commandchars=\\\{\}}
__rawins \{
	\textit{insdata} ;
	\textit{insdata} ;
	...
\}
\end{Verbatim}

每条\textit{insdata}为:

\verb|{| \textit{id difficulty_mask pop_count param_count param_mask, param1 param2 ...} \verb|}|

\end{document}
